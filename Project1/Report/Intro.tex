\documentclass[Report.tex]{subfiles}
\begin{document}
\section{Introduction}
The following paper seeks to investigate Cryptography issues in
within the domain of embedded systems. In specific this report will conduct
three individual performance investigations of the AES-128 encryption environment 
on a ATmega16 micro controller. The dimensions of this analysis will be, RAM consumption,
Code Size, and finally Execution Time, all of which will be implemented using the low level 
programming language C, and analysed using AVR Studio 6. \\
\textit{Knowledge about AES and Cryptography in general is assumed, 
as well as basic programming skills in C. However, this article will provide a short introduction
towards AES-128 in terms of algorithmic specifications.}

\subsection{AES-128 - Algorithm Summary}\label{ALGsum}
As mentioned the AES cryptography system that this 
article will focun on is the AES-128 environment with 10 rounds.
The operations of AES and their order is given in table~\ref{GEAES}.\\\\
In short terms, the \verb+AddRoundKey+ operation simply XOR's the message
with the key from the key schedule.\\
The Substitution Bytes operation simply substitutes the message value with the value
of the SBox table. That is the value of the message byte becomes the index.\\
Shiftrows operates by shifting every row $x$ positions to the left, where $x$ is the row number.
Eg: row 0 stays the same, row 1 is shifted 1 position to the left, and so on.\\
Finally, the slightly more complex Mix Columns operation, works by multiplying the matrix M
with the each column such that the keep their order.

\begin{table}[h]
\begin{center}
    \begin{tabular}{|l|l|}
    \hline
    AES Encryption over n rounds & ~               \\ \hline
    AddRoundKey                  & Pre-whitening   \\ \hline
    SubBytes                     & ~               \\
    ShiftRows                    & For n-1 rounds  \\
    MixColumns                   & ~               \\
    AddRoundKey                  & ~               \\ \hline
    SubBytes                     & ~               \\
    ShiftRows                    & The Final Round \\
    AddRoundKey                  & ~               \\ \hline
    \end{tabular}
    \caption{\label{GEAES}The general AES algorithm structure}
\end{center}
\end{table}

\end{document}