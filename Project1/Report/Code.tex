\documentclass[Report.tex]{subfiles}
\begin{document}
\section{AES-128 optimized for Code Size}
The following section will illustrate tricks and exemplification 
upon how to optimise C-Code in the dimension of Code-Size.
That is, by minimising the Program Memory consumption.
Again, this optimisation is done on the AES-128 Cryptography 
Algorithm.

\subsection{Summarise}
As observed it was possible to do an astonishing $(1-(1888/2594))\cdot100 = 27\%$ reduction in
program memory compared to the Speed optimisation. However, this comes with a trade off.
From table~\ref{OPTprogram} it is readily seen, that the clock count of this optimisation strategy
suffered great impact. If we compare this result to the speed-dimension, an $45490/21499 = 2.12$
factor difference, meaning that this programme executes more than twice as slow compared to
the speed optimisation. Moreover, it is noticed that the RAM consumption merely changes between
these two implementation strategies.
 
\begin{table}[h]
\centering
    \begin{tabular}{|l|l|l|}
    \hline
    Optimization dimention 	& Value       			& Configuration Level 	\\ \hline
    Program Memory         	& 1888 Bytes   		& -O0                 		\\ \hline
    RAM Consumption        	& 350 Bytes    		& -O0                 		\\ \hline
    Clock Cycles           		& 45490 Clocks 		& -O0                 			\\ \hline
    \end{tabular}
    \caption{\label{OPTprogram} Optimisation for Program Memory on an ATmega16 Micro Controller using -O0 configuration}
\end{table}

\subsection{Round key generation}
As for the previous implementation, the round key generation of this
strategy consists of two parts, see below. Moreover,
the function takes the round number as input parameter.
The implementation is illustrated in figure~\ref{RKcode}.
From the figure, it is seen how this approach utilises the 
usage of loops in order to achieve the same functionality. This is done,
since loops reduces code size, both visually and in terms of program memory. 

\begin{itemize}
\item[1] If round number is $0$ then set round key equal to the key
\item[2] If round number is not $0$ then modify the previous round key according to Algorithm~\ref{RK_algorithm}.
\end{itemize}

\begin{figure}[h]
\begin{lstlisting}[basicstyle=\tiny]
static void roundKeyGeneration(uint8_t i)
{
  if(i == 0)
  {
    for(int i = 0; i < 16; i++)
    {
      roundKey[i] = key[i];
    }
  }
  else
  {
    uint8_t temp[4] =
    {
      sBox[roundKey[13]] ^ rCon[i], sBox[roundKey[14]], sBox[roundKey[15]], sBox[roundKey[12]]
    };
    
    for(int i = 0; i < len; i++)
    {
      roundKey[i] ^= (i < 4) ? temp[i] : roundKey[i-4];
    }
  }
}
\end{lstlisting}
\caption{\label{RKcode} Round key implementation optimized for speed.}
\end{figure}

\subsection{Pre-whitening}
It really becomes visual how loops reduces
the code size, when an example like the pre-whitening implementation below, figure~\ref{PWcode},
is provided. What previously consisted of $16$ lines of code is combined
into only $4$ lines, thus removing $1/4$ of the visual complexity.
Furthermore, from the implementation, line 4, it is 
seen that the pre-whitening procedure is used to load
the input message into our global static AES state variable.  

\begin{figure}[h]
\begin{lstlisting}[basicstyle=\tiny]
roundKeyGeneration(round);
for (int i = 0; i < len; i++)
{
  message[i] = in_message[i] ^ roundKey[i];
}
\end{lstlisting}
\caption{\label{PWcode} Round key implementation optimized for program memory.}
\end{figure}

\subsection{The n-1 rounds}
From figure~\ref{N1code} the implementation of the AES-128,
n-1 round iterations is illustrated. Again, a loop has been 
utilized to minimise visual and memory complexity. Moreover,
a observation is made in this implementation. 
If the strategy did not utilise the \verb+temp[4]+ byte array to store the
result from the substitution and shift row operation, and thus simply
recalculated them every time in-line in the Mix Column operations,
the programme memory is increased to $2660$ bytes, which is even more
than the space optimisation implementation. Finally, it is worthwhile
noting that the encryption of the very message and the round key,
in this strategy, is provided in its own function, figure~\ref{AKcode}.

\begin{figure}[h]
\begin{lstlisting}[basicstyle=\tiny]
while(round < 10)
{
  for(int i = 0; i < 4; i++)
  {
  	// Subbytes + Shiftrows + MixColumns
    temp[0] = sBox[message[(i*4) % 16]];
    temp[1] = sBox[message[(i*4+5) % 16]];
    temp[1] = sBox[message[(i*4+10) % 16]];
    temp[1] = sBox[message[(i*4+15) % 16]];			
    temporaryState[i*4] = (x2(temp[0]) + (x2(temp[1]) ^ temp[1]) + temp[1] + temp[1]);
    temporaryState[i*4+1] = (temp[0] + x2(temp[1]) + (x2(temp[1]) ^ temp[1]) + temp[1]);
    temporaryState[i*4+2] = (temp[0] + sBox[temp[1]] + x2(temp[1]) + (x2(temp[1]) ^ temp[1]));
    temporaryState[i*4+3] = ((x2(temp[0]) ^ temp[0]) + temp[1] + temp[1] + x2(temp[1]));
  }	
  // AddRowKey
  roundKeyGeneration(round);
  encrypt();
  round = round + 1;
}
\end{lstlisting}
\caption{\label{N1code} The n-1, 9, rounds of subbytes, shiftrows, mixcolumns, and addkey operations of AES-128, optimised for code size.}
\end{figure}

\begin{figure}[h]
\begin{lstlisting}[basicstyle=\tiny]
static void encrypt()
{
  for (int i = 0; i < len; i++)
  {
    message[i] = temporaryState[i] ^ roundKey[i];
  }
}
\end{lstlisting}
\caption{\label{AKcode} The Add Key implementation of the AES-128, optimised for code size.}
\end{figure}

\subsection{The final round}
Figure~\ref{FRcode}, provides the strategy of the 
final round, that is the 10th round of the algorithm.
It is readily seen, that the implementation utilises a loop
in order to minimise the code complexity. In addition, this loop
iterates over four bytes at a time. This implementation is chosen
since it allows for a small speed optimisation. In specific this 
trick allows for us to do the shiftrows and the byte substitution operation 
concurrently.

\begin{figure}[h]
\begin{lstlisting}[basicstyle=\tiny]
// Subbytes + Shiftrows
for(int i = 0; i < 4; i++)
{
  // Subbytes + Shiftrows
  temporaryState[i*4] = sBox[message[(i*4) % 16]];
  temporaryState[i*4+1] = sBox[message[(i*4+5) % 16]];
  temporaryState[i*4+2] = sBox[message[(i*4+10) % 16]];
  temporaryState[i*4+3] = sBox[message[(i*4+15) % 16]];
}
// AddRoundKey
roundKeyGeneration(round);
encrypt();
\end{lstlisting}
\caption{\label{FRcode} The final round, optimised for code size.}
\end{figure}

\end{document}

