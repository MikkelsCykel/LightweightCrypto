\documentclass[Report.tex]{subfiles}
\begin{document}
\section{AES-128 optimized for Ram Consumption}
The final dimension of this  study, is concerning 
minimising of RAM memory. This implementation
is based upon the speed implementation in the beginning of this article.
The difference however it to find in the introduction of FLASH Memory.
That is in this exercise, the huge S-Box table used in the substitution bytes operation,
will be moved to the FLASH memory, by using the \verb+PROGMEM+ interface. This
is illustrated in figure~\ref{SBOXFLASH}.


\subsection{Summarise}
By storing the S-Box data table in the flash memory of the 
ATmega16 processor, the total ram size needed is reduced
$(1-(60/350))\cdot = 82\%$, thus really minimising the ram consumption.
But as expected this is slower than having the data available in the RAM.
And from table~\ref{OPTram} this reduce is really visible. Since the RAM
optimisation is directly based on the speed implementation, we denote that
by introducing FLASH memory, the execution time is growing a little less
than 8000 clocks. Moreover, it is noticed that the Program Memory needed in
this implementation grew approximately by a factor $5780/1888 = 3$, which is again expected
since it the code now need interaction with the FLASH via the BUS.

\begin{table}[h]
\centering
    \begin{tabular}{|l|l|l|}
    \hline
    Optimization dimention 	& Value       			& Configuration Level 	\\ \hline
    Program Memory         	& 5780 Bytes   		& -O0                 		\\ \hline
    RAM Consumption        	& 60 Bytes    		& -O0                 		\\ \hline
    Clock Cycles           		& 29355 Clocks 		& -O0                 			\\ \hline
    \end{tabular}
    \caption{\label{OPTram} Optimisation for speed on an ATmega16 Micro Controller using -O0 configuration}
\end{table}

\begin{figure}[h]
\begin{lstlisting}[basicstyle=\tiny]
const uint8_t sBox[256] PROGMEM=
{
	0x63, 0x7c, 0x77, 0x7b, 0xf2, 0x6b, 0x6f, 0xc5,
	0x30, 0x01, 0x67, 0x2b, 0xfe, 0xd7, 0xab, 0x76,
	0xca, 0x82, 0xc9, 0x7d, 0xfa, 0x59, 0x47, 0xf0,
	0xad, 0xd4, 0xa2, 0xaf, 0x9c, 0xa4, 0x72, 0xc0,
	0xb7, 0xfd, 0x93, 0x26, 0x36, 0x3f, 0xf7, 0xcc,
	0x34, 0xa5, 0xe5, 0xf1, 0x71, 0xd8, 0x31, 0x15,
	0x04, 0xc7, 0x23, 0xc3, 0x18, 0x96, 0x05, 0x9a,
	0x07, 0x12, 0x80, 0xe2, 0xeb, 0x27, 0xb2, 0x75,
	0x09, 0x83, 0x2c, 0x1a, 0x1b, 0x6e, 0x5a, 0xa0,
	0x52, 0x3b, 0xd6, 0xb3, 0x29, 0xe3, 0x2f, 0x84,
	0x53, 0xd1, 0x00, 0xed, 0x20, 0xfc, 0xb1, 0x5b,
	0x6a, 0xcb, 0xbe, 0x39, 0x4a, 0x4c, 0x58, 0xcf,
	0xd0, 0xef, 0xaa, 0xfb, 0x43, 0x4d, 0x33, 0x85,
	0x45, 0xf9, 0x02, 0x7f, 0x50, 0x3c, 0x9f, 0xa8,
	0x51, 0xa3, 0x40, 0x8f, 0x92, 0x9d, 0x38, 0xf5,
	0xbc, 0xb6, 0xda, 0x21, 0x10, 0xff, 0xf3, 0xd2,
	0xcd, 0x0c, 0x13, 0xec, 0x5f, 0x97, 0x44, 0x17,
	0xc4, 0xa7, 0x7e, 0x3d, 0x64, 0x5d, 0x19, 0x73,
	0x60, 0x81, 0x4f, 0xdc, 0x22, 0x2a, 0x90, 0x88,
	0x46, 0xee, 0xb8, 0x14, 0xde, 0x5e, 0x0b, 0xdb,
	0xe0, 0x32, 0x3a, 0x0a, 0x49, 0x06, 0x24, 0x5c,
	0xc2, 0xd3, 0xac, 0x62, 0x91, 0x95, 0xe4, 0x79,
	0xe7, 0xc8, 0x37, 0x6d, 0x8d, 0xd5, 0x4e, 0xa9,
	0x6c, 0x56, 0xf4, 0xea, 0x65, 0x7a, 0xae, 0x08,
	0xba, 0x78, 0x25, 0x2e, 0x1c, 0xa6, 0xb4, 0xc6,
	0xe8, 0xdd, 0x74, 0x1f, 0x4b, 0xbd, 0x8b, 0x8a,
	0x70, 0x3e, 0xb5, 0x66, 0x48, 0x03, 0xf6, 0x0e,
	0x61, 0x35, 0x57, 0xb9, 0x86, 0xc1, 0x1d, 0x9e,
	0xe1, 0xf8, 0x98, 0x11, 0x69, 0xd9, 0x8e, 0x94,
	0x9b, 0x1e, 0x87, 0xe9, 0xce, 0x55, 0x28, 0xdf,
	0x8c, 0xa1, 0x89, 0x0d, 0xbf, 0xe6, 0x42, 0x68,
	0x41, 0x99, 0x2d, 0x0f, 0xb0, 0x54, 0xbb, 0x16
};
\end{lstlisting}
\caption{\label{SBOXFLASH} The S-Box is stored in the Flash Memory.}
\end{figure}

\end{document}